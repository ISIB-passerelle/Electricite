\documentclass[]{article}
\usepackage{lmodern}
\usepackage{amssymb,amsmath}
\usepackage{ifxetex,ifluatex}
\usepackage{fixltx2e} % provides \textsubscript
\usepackage{xcolor}
\usepackage[margin=2.5cm]{geometry}
\ifnum 0\ifxetex 1\fi\ifluatex 1\fi=0 % if pdftex
  \usepackage[T1]{fontenc}
  \usepackage[utf8]{inputenc}
\else % if luatex or xelatex
  \ifxetex
    \usepackage{mathspec}
  \else
    \usepackage{fontspec}
  \fi
  \defaultfontfeatures{Ligatures=TeX,Scale=MatchLowercase}
\fi
% use upquote if available, for straight quotes in verbatim environments
\IfFileExists{upquote.sty}{\usepackage{upquote}}{}
% use microtype if available
\IfFileExists{microtype.sty}{%
\usepackage{microtype}
\UseMicrotypeSet[protrusion]{basicmath} % disable protrusion for tt fonts
}{}
\usepackage{hyperref}
\hypersetup{unicode=true,
            pdfborder={0 0 0},
            breaklinks=true}
\urlstyle{same}  % don’t use monospace font for urls
\IfFileExists{parskip.sty}{%
\usepackage{parskip}
}{% else
\setlength{\parindent}{0pt}
\setlength{\parskip}{6pt plus 2pt minus 1pt}
}
\setlength{\emergencystretch}{3em}  % prevent overfull lines
\providecommand{\tightlist}{%
  \setlength{\itemsep}{0pt}\setlength{\parskip}{0pt}}
\setcounter{secnumdepth}{0}
% Redefines (sub)paragraphs to behave more like sections
\ifx\paragraph\undefined\else
\let\oldparagraph\paragraph
\renewcommand{\paragraph}[1]{\oldparagraph{#1}\mbox{}}
\fi
\ifx\subparagraph\undefined\else
\let\oldsubparagraph\subparagraph
\renewcommand{\subparagraph}[1]{\oldsubparagraph{#1}\mbox{}}
\fi

\renewcommand{\familydefault}{\sfdefault}


\date{}

\begin{document}

\section{Électricité}\label{electricite}

\subsection{Formules}\label{formules}



\paragraph{Courant électrique}\label{courant-electrique}

{\large $$ I = \dfrac{U}{R} $$ }


\begin{itemize}
	\item 
		$ I $ en \textbf{ampères} ($A$)
	\item
		$ U $ en \textbf{volts} ($V$)
	\item
		$ R $ en \textbf{ohms} ($\Omega$)
\end{itemize}



\paragraph{Tension}\label{tension}


{\large $$ U = R \cdot I $$ }

\begin{itemize}
	\item 
		$ U $ en \textbf{volts} ($V$)
	\item
		$ I $ en \textbf{ampères} ($A$)
	\item
		$ R $ en \textbf{ohms} ($\Omega$)
\end{itemize}



\paragraph{Résistance}\label{resistance}

{\large $$ R = \dfrac{U}{I} $$ }

\begin{itemize}
	\item 
		$ R $ en \textbf{ohms} ($\Omega$)
	\item
		$ U $ en \textbf{volts} ($V$)
	\item
		$ I $ en \textbf{ampères} ($A$)
\end{itemize}



\paragraph{Puissance}\label{puissance}

{\large $$ P = U \cdot I $$ }

\begin{itemize}
	\item 
		$ P $ en \textbf{watts} ($W$)
	\item
		$ U $ en \textbf{volts} ($V$)
	\item
		$ I $ en \textbf{ampères} ($A$)
\end{itemize}



\paragraph{Puissance électrique transformée en chaleur (effet Joule)}\label{puissance-chaleur}

{\large $$ P = U \cdot I = (R \cdot I) \cdot I = R \cdot I^2 = \dfrac{U^2}{R}$$ }

\begin{itemize}
	\item 
		$ P $ en \textbf{watts} ($W$)
	\item
		$ U $ en \textbf{volts} ($V$)
	\item
		$ I $ en \textbf{ampères} ($A$)
	\item 
		$ R $ en \textbf{ohms} ($\Omega$)
\end{itemize}



\paragraph{Énergie électrique}\label{energie-electrique}

{\large $$ W = P \cdot t $$ }

\begin{itemize}
	\item 
		$ W $ en \textbf{joules} ($J$)
	\item
		$ P $ en \textbf{watts} ($W$)
	\item
		$ t $ en \textbf{secondes} ($s$)
\end{itemize}



\subsection{Propriétés des circuits en série}\label{propriete-circuit-serie}

\begin{enumerate}
	\item 
		Le courant est le même dans tous les éléments du circuit.
		
		$$ I_M = I_L = I_C = \dots $$
	\item
		La somme des tensions aux bornes des charges est égale à la tension aux bornes de la source.
		
		$$ U_T = U_1 + U_2 + U_3 + \dots $$
	\item
		La somme des puissances absorbées par les charges est égale à la puissance fournie par la source.
		
		$$ P_T = P_1 + P_2 + P_3 + \dots $$
		
\end{enumerate}


\subsubsection{Résistance équivalente dans un montage en série}\label{resistance-equivalente}

La résistance de l’ensemble des résistances d’un circuit en série est égale à la somme des résistances individuelles.

$$ R_{équivalente} = R_1 + R_2 + R_3 + \dots $$



\subsection{Propriétés des circuits en parallèle}\label{propriete-circuit-parallele}

\begin{enumerate}
	\item 
	La tension est la même aux bornes de chaque élément.
	
	$$ U_M = U_L = U_C = \dots $$
	\item
	La somme des courants tirés par les charges est égale au courant débité par la source.
	
	$$ I_T = I_1 + I_2 + I_3 + \dots $$
	\item
	La somme des puissances consommées par les charges est égale à la puissance fournie par la source.
	
	$$ P_T = P_1 + P_2 + P_3 + \dots $$
	
\end{enumerate}


\subsubsection{Résistance équivalente dans un montage en parallèle}\label{resistance-equivalente-parallele}

La résistante unique équivalente à deux résistances en parallèle est égale au produit des résistances divisés par leur somme.

$$ R_{équivalente} = \dfrac{R_1 \cdot R_2}{R_1 + R_2} $$


\subsection{Circuits à courant continu}\label{circuit-courant-continu}


\paragraph{Première loi de Kirchhoff}\label{1re-loi-kirchhoff}

La somme algébrique des tensions dans une boucle fermée d’un circuit est égale à zéro.


\paragraph{Deuxième loi de Kirchhoff}\label{2e-loi-kirchhoff}

La somme des courants qui arrivent à un nœud est égale à la somme des courants qui en partent.


\paragraph{Énoncé du théorème de Thévenin}

Tout circuit à deux bornes ouvertes $A$ et $B$ composé de plusieurs sources et de plusieurs résistances peut être remplacé par une source unique $E$ en série avec une résistance unique $R$.

\paragraph{Théorème de superposition}

Le courant circulant dans un élément de circuit est égal à la somme algébrique des courants qui seraient produits dans cet élément par chacune des sources agissant seule, les autres sources étant remplacées par des court-circuits.


\end{document}
